\documentclass [12pt, letterpaper] {article}

\usepackage{amsmath}
\usepackage{amssymb}
\usepackage{amsthm}
\usepackage{fancyvrb}
\usepackage{../../cs3320}
\usepackage{hyperref}

\begin{document}
\project{1}{Tuesday, Sept. 28, 2021}
\progdirections
In this project you will gain experience with Linux system calls by
constructing a new Linux shell. Your shell will be called msh (Merrimack
Shell). Recall that a shell is a command line interface (CLI) that allows
you to interact with the operating system.
The assignment will be broken down into phases to help you
organize your thoughts. To help you get started I have uploaded starter
code to classroom.

Notice you have been given \textbf{two(2) weeks} to complete this
project. This project can \textbf{not} be completed in one night.
You should complete the phases below in order. Each phase will consists
of a few days work (see recommend completion date).

\section*{Phase I -- Basic Shell (\emph{35 points})}
\noindent\textbf{Recommended Completion Date}: Friday, Sept. 21\\
\\
In the first phase of this project you will implement the majority of the
shell. You must construct the basic command line interpreter (it's the classic REPL algorithm
we discussed in class). Your shell
(i.e., the main process) should enter a tight loop that prompts the user
for a command and then executes that command by {\tt fork(2)}-ing the main
process and having the child {\tt exec(3)} the command. The parent should
use {\tt wait(2)} to wait for the completion of the child process before
continuing.

There is one exception to the program flow above. If the user enters the
command {\tt quit} or {\tt exit}, the main process should not call {\tt fork(2)} and
instead just exit the program successfully.

An example interaction with my solution to Phase I is below. Please make
your output look \emph{exactly} like mine. In other words, please make
your prompt be {\tt msh>}. You can just call \statement|get_command| in
\texttt{shell.c} to achieve this task.

\begin{Verbatim}[frame=single]
 msh> make clobber
 rm -f shell.o history.o
 rm -f shell
 msh> ls
 head  history.cpp  Makefile  shell.cpp
 msh>
 msh>
 msh> quit
\end{Verbatim}
Notice that I may hit enter as many times as I want with out causing the
shell to do any work.

As mentioned previously, you will need the {\tt fork(2)} and {\tt exec(3)}
system calls. Recall that {\tt exec(3)} is a family of system of calls. I
would recommend you use the {\tt execvp(3)} system call. To do use this you
will need to construct an array of C strings (character arrays terminated
with a \statement|NULL| character). This special array you construct is sometimes
referred to as the {\tt argv} array. The first element in the array
is the name of the command and the subsequent elements are are the options
to the command. To help you with converting a C string to
this \statement|argv| array, I have provided you with the the function
\statement|build_argument_array| in {\tt shell.c}.

The manual pages (man pages) will provide some assistance with {\tt fork(2)}
and {\tt execvp(3)}.

\section*{Phase II -- Backgrounding Processes (\emph{30 points})}
\noindent\textbf{Recommended Completion Date}: Thursday, Sept. 23\\
\\
In this phase your are going to extend your shell with the ability to
launch what is called a background process. To launch a background process
the user appends and ampersand(\&) to the end of the command

A background process is run in a child process just like in Phase I,
except with a background process the shell won't wait for the process.
Won't this make zombies you ask? Why yes it will if you don't keep track
of the outstanding process and make sure they are all finished before you
exit your program.
Recall that when you {\tt fork(2)} a process {\tt fork(2)} returns,
to the parent process, a non-zero number. This non-zero number is actually
the PID of the newly constructed child process. After the parent process
has {\tt fork(2)}-ed its child process you should add this new PID to a
linked list. You can use the \texttt{joblist} data structure I have provided
for you to keep track of the active PIDs.

You must make some additional modifications as well. You must:
\begin{itemize}
 \item Modify the parent code, following the {\tt fork(2)} call, to add
       the PID to the list of outstanding  PIDs if the command is to be
       backgrounded. If it is not a background command you should wait for the
       process to complete. There is a pitfall here, when you call {\tt wait(2)}
       the call will return once one of the children has completed. The return
       value will tell you which one. Please make sure you wait for the correct
       PID. If a different PID completes you should remove that PID from the list
       of PIDs.
 \item Modify your code that handles {\tt quit} and {\tt exit} to make sure that all
       the processes in the list have completed (i.e., the list is empty). If
       the list is not empty, you should call {\tt wait(2)} until it is empty. Note
       that {\tt wait(2)} returns -1 and sets {\tt errno} to {\tt ECHILD} if there
       are no children of the process left. If this occurs, you can just empty
       the list and exit the program. The {\tt errno} variable is a special
       global variable used to pass errors around Unix based systems. To use
       this global just include {\tt errno.h} in your program.
\end{itemize}

The functions that are provided by a the {\tt joblist} data structure can be
found in \texttt{head/joblist.h}. In order to understand what each function does,
you should read the header comments in {\tt joblist.c}.
%To help with the linked list I would use the STL's {\tt list} class. For
%a list of methods available to you in the {\tt list} class please see
%\url{http://www.cplusplus.com/reference/list/list/}

\section*{Phase III -- Add Built-In Commands (\emph{20 points})}
\noindent\textbf{Recommended Completion Date}: Monday, Sept. 27\\
\\
In this phase you will add some additional built-in commands. Built-in
commands are commands that are part of the shell program itself. We are
going to add command line history to our shell. To help you with this
process I have provided you with a finite buffer class. I would
recommend you use it to manage your command line history.

You should add the following commands:
\begin{itemize}
 \item {\tt history}: This command will print out the current contents of
       the history table or message indicating the history is empty. This
       command should itself be added to the history buffer.
 \item {\tt !!}: This command will execute the last command the user
       executed. This command should not be added to the history buffer.
 \item {\tt !$n$}: This command tells the shell to execute the $n$-th
       command in the history. If no such command exists it should display an
       error. This command should not be added to the history buffer. You may
       find the function {\tt atoi(3)} helpful in parsing this command.
 \item {\tt cd }\emph{dir}: This command changes the directory the the location
       given by \emph{dir}. This is very similar to the \texttt{cd} built-in command in
       BASH (e.g., {\tt man cd}). You do \emph{not} have to implement {\tt cd} without a directory
       or \verb#cd ~#. You will need to use the \texttt{chdir(2)} system call.
\end{itemize}
Unless otherwise noted, every command should be added to the history
buffer. You will find the {\tt history} data structure in {\tt history.h} and
 {\tt history.c} helpful in your task. Your finite buffer should be initialized
to hold 5 commands in the history.

\section*{Style (\emph{10 points})}
Please be sure to use good function decomposition and follow the department standard when coding your
solution. Make sure your code is well commented I need to understand
what you understand. If you have any style questions please ask.

\section*{A Clean Build (\emph{5 points})}
To build your program you just need to type {\tt make(1)} at the command prompt. The file {\tt Makefile} has comments
if you are curious how {\tt Makefile}'s go together. Your code should:
\begin{itemize}
 \item Compile clean (i.e., free of errors and warnings) on Fedora 34 using the command
 {\tt make all}
 \item Contain no memory leaks per \texttt{valgrind}. I will be using \texttt{valgrind} with the
       \texttt{--leak-check=full} when checking your programs.
       \begin{itemize}
         \item You can install \texttt{valgrind} using the command
         \verb|sudo dnf install valgrind|.
       \end{itemize}
\end{itemize}

\section*{Make Targets}
The operations that \texttt{make(1)} takes are called the targets. The makefile
I have provided you with supports three targets of interest:
\begin{enumerate}
  \item {\tt all} this target builds the entire project, resulting in an executable
  called {\tt shell}.
  \item {\tt clean} this target removes all of the intermediate files generated by the
  compiler.
  \item {\tt clobber} this target does everything {\tt clean} does as well as
  deleting the executable {\tt shell}.
\end{enumerate}
It should be noted that {\tt all} is the default target and thus, you can simply
write {\tt make} and the {\tt all} target will run.

\section*{General Reminders}
\begin{itemize}
\item \textbf{Start early}.

\item Each phase comes with a suggested completion date to keep you on track
during the two weeks. You will notice that at least 24 hours of time is recommended to
be used for testing and debugging your project solution.

\item Complete all of the other phases before attempting the extra credit.

\item You should test your shell thoroughly.

\item Make sure to check your code against the assignment \emph{before} submission.

\item You are encouraged to see me for help with the project.

\item You can discuss the project with each other but, the write-up should
be your own.

\item Don't forget about the man pages, they are as helpful as JavaDocs

\item Do not modify the code in functions, unless I told you to.
\end{itemize}


\section*{Submission}
Submissions in this class come in two parts:
\begin{enumerate}
  \item \textbf{Digital Submission}: Since our projects now have multiple files
  that need to be graded, I ask that you submit a zip file containing the \texttt{make}
  file and all associated with headers and source files.
  This zip file \textbf{must be} named with your last name followed by the project number.  Failure to follow
  these directions will result in a deduction of up to \textbf{5 points from your grade}.
  \item \textbf{Hard Copy of Code}: A hard copy of all header and source code
  files you authored for a project. This hard copy should be stapled together as
  one packet. Failure to follow this direction will result in a deduction of
  up to \textbf{5 points from your grade}.
\end{enumerate}
The directions above are for the purpose of ensuring a timely and correct
grading of your project.

\section*{Grading}
To recap the above, your grade on this assignment will be determined as follows:
\begin{itemize}
 \item Program has good style and function decomposition (\emph{10 points}).
 \item Program correctly implements phase I (\emph{35 points}).
 \item Program correctly implements phase II (\emph{30 points}).
 \item Program correctly implements phase III (\emph{20 points}).
 \item Program compiles clean and has no memory leaks (\emph{5 points}).
 \item Correctly implemented Extra credit I (\emph{5 points}).
 \item Correctly implemented Extra credit II (\emph{5 points}).
\end{itemize}

\section*{Extra Credit I (\emph{5 points})}
You can extend your implementation of the \texttt{cd} built-in command to support
\texttt{cd} without a path after. This invocation, returns the user
to their home directory. You can experiment with this behavior on your VM. To implement
this feature, you should you will need to
read the \verb#HOME# environment variable. You should use \texttt{apropos(1)} to
determine what function (hint it will be in section 3) will allow you to read the
\verb#HOME# environment variable. For example:
\texttt{cd} is equivalent to \texttt{cd /home/kisselz} when I executed the command on my machine.

\section*{Extra Credit II (\emph{5 points})}
Once you have implemented the first extra credit, you can extend \texttt{cd} again. In this
extension, you will handle \verb#~#. When you see a \verb#~# at the start of a path provided to
\texttt{cd} you should replace \verb#~# with the contents of the \verb#HOME# environment variable.
For example:
\begin{itemize}
 \item \verb#cd ~# is equivalent in behavior to \texttt{cd}
 \item \verb#cd ~# is equivalent in behavior to \texttt{cd /home/kisselz} for myself on Fedora 34.
 \item \verb#cd ~/public/os-examples# is equivalent in behavior to \texttt{cd /home/kisselz/public/os-examples} when
       I execute the command on Fedora 34.
\end{itemize}
Again, you will have to use \texttt{apropos(1)} to determine what function call will allow you to read
the \verb#HOME# environment variable.
\end{document}
